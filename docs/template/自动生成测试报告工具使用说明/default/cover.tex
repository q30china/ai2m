% ======== 标题 ==============================

%\title{论文标题(中文)}

%\author{作者姓名\thanks{单位信息\quad 邮箱:abc@example.com}}
%\date{2025 年 8 月 12 日}
%\maketitle
%\newpage

% 封面
% 里通常只包含标题页、作者、机构、日期等信息,常用

% ======== 1.Cover页,第一页,标题 ==============================

% 1. 在文档中设置标题、作者
\title{测试报告数据自动映射工具\\使用说明 \\ \small{v. 1.0.0}}
\author{Qiao Lei}

\thispagestyle{fancy}

% 2. 在 \begin{document} 后调用 \maketitle
\maketitle

\clearpage

% ======== 2. 摘要 ==============================
%\renewcommand{\abstractname}{摘要}
%\renewcommand{\abstractnamefont}{\fontsize{14}{16}\selectfont}
% 可以加载abstract包,或者直接用下面标准类,改变字体
\renewcommand{\abstractname}{\Large\bfseries 摘要}

\pagenumbering{arabic}
% 当前页从page2 开始
\setcounter{page}{2}

\begin{abstract}
测试中心负责对事业部委托的智能电表样品进行性能与可靠性评估,并依据 ISO/IEC 17025 资质标准出具正式实验室认可报告。
在以往的操作流程中,测试人员需将大量测量数据,尤其是基本误差数据,手工录入报告模板,既耗时又容易发生疏漏。为克服这一瓶颈,我们研发了一款自动化数据填报工具,能够直接将测量系统输出的数据导入并完成报告的填充,显著提升工作效率与数据准确性。本篇文章将系统介绍该工具的功能、使用方法,并在必要时提供二次开发的指导。
\end{abstract}

% ======== 3. 关键词 ==============================

\begin{keywords}

测试,报告,自动化

\end{keywords}

\clearpage

% ======== 4. 自动生成目录 ==============================
\tableofcontents
\clearpage