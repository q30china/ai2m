% ========================================================
%  中文写作 LaTeX 模版(XeLaTeX)
%  适合论文、报告、技术文档、书籍等
%  ========================================================

% --------- 1. 文档类与编码 ------------------------------------
\setlength{\parskip}{1.2em}  %em是当前字体大小的单位,设置这个后,会取消默认的段落首行缩进

\setlength{\parindent}{0pt}            % 这里是首行缩进的设置,若想取消,打开注释即可

% --------- 2. 基本宏包 ----------------------------------------
\usepackage{xeCJK}        % 中文支持(XeLaTeX)
\usepackage{fontspec}     % 直接使用系统字体
\usepackage{geometry}     % 页面边距和版式
\usepackage{setspace}     % 行距,可使用\doublespacing 设置双倍行距
\usepackage{titlesec}     % 自定义章节标题,后面会配置section和subsection的格式
\usepackage{fancyhdr}     % 页眉页脚自定义
\usepackage{hyperref}     % 生成 PDF 超链接,自动为目录,引用等生成超链接,为pdf添加交互功能
\usepackage{graphicx}     % 插入图片,支持pdf,png,jpg
\usepackage{caption}      % 调整图表标题格式
\usepackage{natbib}       % 参考文献管理
\usepackage{amsmath,amssymb} % 数学公式,AMS数学符号和公式环境
\usepackage{subcaption}   % 子图环境,可以创建带有(a)、(b)等标签的子图
\usepackage{listings}     % 代码高亮,这里主要使用bash等命令的高亮,其它的如python,html还是用minted
\usepackage{xcolor}       % 颜色
%\usepackage{array}       % keywords那里使用,提供表格增强包
\usepackage{booktabs}     % toprule\botomrule\midrule,专业质量的表格线
\usepackage{minted}       % 代码高亮显示,比listings包更强大
\usepackage{multirow}     % table,表格的多行合并功能
\usepackage{fontawesome5} % 提供Font Awesome图标, 比如页脚faPhone
\usepackage{lastpage}     % lastpage,获取总页数
%\usepackage{abstract}    % abstract,摘要环境
\usepackage{indentfirst}  % first line,确保每个章节的第一段首行缩进
%\usepackage{ctex}        % ctex宏包是LaTeX中处理中文文档的核心工具之一,它提供了全面的中文排版支持,简化了中文文档的制作流程。这里不要打开,会产生报警。自己控制即可。

% --------- 3. 页面设置(大小、边距) --------------------------
% \geometry{a4paper, headheight=18pt, left=30mm, right=30mm, top=30mm, bottom=35mm}

\geometry{a4paper, headheight=18pt, left=25mm, right=25mm, top=22mm, bottom=22mm}
% headheight 页眉高度18pt, 左右边距25,上下边距22

\doublespacing            % 行距 1.5 倍行距

% --------- 4. 中文字体 ---------------------------------------
% 默认中文字体(SimSun)可以换成你系统中存在的字体
% 例如:\setCJKmainfont[BoldFont=SimHei]{Microsoft YaHei}

% 在overleaf上使用,建议使用开源字体
% 文泉驿系列字体(WenQuanYi)
% 举例: \setCJKmainfont[BoldFont=WenQuanYi Zen Hei Bold]{WenQuanYi Micro Hei}  Linux平台默认开源黑体

% 思源系列字体(Source Han) 思源黑体‌(Source Han Sans),思源宋体‌(Source Han Serif)
% Noto	Google 的字体项目名称,旨在支持全球所有语言,避免“豆腐块”(□)字符。
% Sans	表示“无衬线”(sans-serif),字体线条简洁、现代,没有额外的装饰笔画。
% Mono	表示“等宽”(monospaced),每个字符占据相同的水平空间,适合代码排版。
% CJK	指 Chinese, Japanese, Korean(中日韩),表示该字体支持中日韩统一表意文字。
% SC	是 Simplified Chinese 的缩写,表示这是简体中文版本。

% \setCJKmainfont{Noto Sans CJK SC}       % 正文字体,Google和Adobe合作开发的开源字体,(思源黑体简体)
% \setCJKsansfont{SimHei}                 % 无衬线字体为黑体
% \setCJKmonofont{Noto Sans Mono CJK SC}  % 等宽字体,

% Mac本地编译时,用下面这组,字体找不到,会提示一堆错误
%\setCJKmainfont{Noto Sans SC}[BoldFont=Hei]       % 正文字体,Google和Adobe合作开发的开源字体,(思源黑体简体)
\setCJKmainfont{宋体-简}       % 正文字体,宋体-简
\setCJKsansfont{黑体-简}                 % 无衬线字体              % 无衬线字体为黑体
\setCJKmonofont{Noto Sans SC}  % 等宽字体,
 
%\usepackage[scheme=plain]{ctex}        % 自动选择字体,ctex这个包没有使用过


% ---------5.英文字体(可选)---------------------------------------
\setmainfont{Times New Roman}                     % 西文字体(英文字母、公式、参考文献等),学术论文常用衬线字体
\setsansfont{Arial}                               % 无衬线英文字体


% --------- 6. 标题、目录、页眉页脚 -----------------------------

\titlespacing*{\section}
    {0pt}    % 左边缩进(通常为 0)
    {0em}    % 标题**前面**的间距(上间距)
    {0.2em}  % 标题**后面**的间距(下间距)

\titlespacing*{\subsection}
    {0pt}    % 左边缩进(通常为 0)
    {0em}    % 标题**前面**的间距(上间距)
    {0.2em}  % 标题**后面**的间距(下间距)
    

\titleformat{\section}
  {\normalfont\Large\bfseries}{\thesection}{1em}{} % 正常字体,Large大小,加粗,编号和标题间距:1em
\titleformat{\subsection}
  {\normalfont\large\bfseries}{\thesubsection}{1em}{}


% 页眉页脚(可自行修改)
%\fancyhf{}
%\fancyhead[L]{这是页眉显示}     % 第一栏:章节标题
%\fancyhead[R]{\thepage}      % 第二栏:页码
%\renewcommand{\headrulewidth}{0.4pt}
%\renewcommand{\footrulewidth}{0pt}
%\pagestyle{fancy}

% 页眉 页脚
\definecolor{customblue}{HTML}{00469C}  % 定义自定义颜色customblue,HTML代码为00469C
% fancy 指的是 fancyhdr 宏包 提供的一种页眉页脚样式,全称是 "fancy page style"。
\pagestyle{fancy}  % 使用fancy页眉页脚样式

\setlength{\headheight}{30.36pt}  % 设置页眉高度40pt,设置太小,低于30.35004pt,编译会有warning
\setlength{\footskip}{15.5pt}    % 设置页脚与正文间距15.5pt. 如果设置30pt,文字和页脚的横线中间留白会很多,最小15.49998
\fancyhf{}  % 清空默认的页眉页脚设置

\fancyhead[L]{\includegraphics[height=20pt]{res/logo.png}}   % 页眉左侧插入图片,高度25pt
% \fancyhead[R]{\color{customblue}\thepage}  % 页眉右侧插入当前页码

\renewcommand{\headrulewidth}{1pt}  % 设置页眉分隔线宽度为1pt
\renewcommand{\footrulewidth}{1pt}  % 设置页脚分隔线宽度为1pt

% 设置页眉页脚分隔线颜色为customblue
\renewcommand{\headrule}{{\color{customblue}\hrule width\headwidth height\headrulewidth}}
\renewcommand{\footrule}{{\color{customblue}\hrule width\headwidth height\footrulewidth}}

% Left part of foot,页脚左侧部分
\fancyfoot[L]{
    \scriptsize   % 设置字体大小为scriptsize(比正文默认字体小一号)
%    Zhejiang Hangzhou  / Moganshan road: 1418-5\\
    % 使用Font Awesome图标(\faPhone)显示电话图标, \quad 插入一个等于当前字体大小的水平间距(约1em)
    % \faGlobe显示地球图标,使用之前定义的customblue颜色,\texttt使用等宽字体显示网址
    \faPhone~+86 15158869595 \quad \faGlobe~\textcolor{customblue}{\underline{\texttt{http://www.xxx.com}}}
}

% Right part of foot,页脚的右侧部分
\fancyfoot[R]{
    \scriptsize
    Page \thepage~of~\pageref{LastPage}
}

% Center part of foot  这个raisebox如果改为-10pt, 则页码page那块上面留白比较大
\fancyfoot[C]{}

\fancyfoot[R]{\raisebox{0pt}{\scriptsize Page \thepage~of~\pageref{LastPage}}}  % 这里垂直位移调整,为了和左侧对齐


% --------- 7. 参考文献风格 ------------------------------------
% plain 的增强版,支持 natbib 宏包,兼容作者-年份和数字引用,推荐!现代文档首选,支持 APA 风格
% apalike,模拟 APA(美国心理学会)格式,作者-年份风格,心理学、社会科学、需要 APA 风格的论文
\bibliographystyle{plainnat}    % 也可以改为 plain / apalike / authoryear 等
% 下面的 .bib 文件在编译时会被调用:\bibliography{references}


%--------- 8. 代码颜色 & 高亮 ,这里用listings 定制shell 命令样式 ---------------------------------
%\lstset{
%  basicstyle=\ttfamily\small,
%  breaklines=true,
%  frame=single,
%  numbers=left,
%  numberstyle=\tiny,
%  keywordstyle=\color{blue},
%  commentstyle=\color{gray},
%  stringstyle=\color{red}
%}

\definecolor{mygreen}{RGB}{0,128,0}
\definecolor{myred}{RGB}{150,0,0}
\definecolor{mygray}{RGB}{128,128,128}

% 全局定义名为ShellStyle的代码样式,后面只需要 \begin{lstlisting} \end{lstlisting} 就可以使用,\begin{lstlisting}[style=OtherStyle],也可以临时使用其它代码样式。
% 语言:bash,关键字样式:洋红色、加粗,注释样式:mygreen颜色,字符串样式:myred颜色,基础样式:小号等宽字体,
% 背景色:mygray的5%透明度, 边框:单线框, 行号:左侧显示, 行号样式:小号、mygray颜色, 制表符大小:4空格, 自动换行:启用,不显示字符串中的空格

\lstdefinestyle{ShellStyle}{
  language=bash,
  keywordstyle=\color{magenta}\bfseries,
  commentstyle=\color{mygreen},
  stringstyle=\color{myred},
  basicstyle=\small\ttfamily,
  backgroundcolor=\color{mygray!5},
  frame=single,
  numbers=left,
  numberstyle=\tiny\color{mygray},
  tabsize=4,
  breaklines=true,
  showstringspaces=false
}
\lstset{style=ShellStyle}


% --------- 9.设置keywords环境 ---------------------------------
% 把关键词包装成一个环境,可以通过 \keywords{LaTeX, 模板, 中文排版} 使用
\newcommand{\keywords}[1]{%
  \par\noindent\textbf{关键词:}\;#1\par
}

% --------- 10.代码块minted代码高亮设置---------------------------------
% 设定行号、框架、字体、背景等
\definecolor{bgcolor}{HTML}{F5F5F5}   % 定义背景色bgcolor为浅灰色
 
\setminted{
    fontsize=\small,
    style=colorful,        % Pygments 提供的样式,如 github、borland、friendly 等
    linenos,               % 显示行号
    numbersep=5pt,         % 行号与代码间距:5pt
    xleftmargin=1em,       % 左边距:1em
    frame=lines,           % 边框:lines样式
    framesep=2mm,          % 边框间距:2mm
    tabsize=4,             % 制表符大小:4空格
    baselinestretch=1.3,   % 行距:1.3倍
    bgcolor=bgcolor        % 需要先用 \definecolor 定义
}

% --------- 11.第一页特殊处理下---------------------------------
% FIRST PAGE
\makeatletter
\renewcommand{\maketitle}{
  \begin{titlepage}
    \thispagestyle{fancy}             % 默认 titlepage 使用 plain(只有页码在底部居中),这里我们改成 fancy,以便可以自定义页眉页脚(比如加 logo、线条、颜色等)。
    \begin{center}
      \vspace*{9cm}                   % 标题下移9cm
      {\LARGE \@title \par}           % Large字体,\par确认换行
      \vspace{1cm}                    % 作者信息下移1cm   
      {\@author \par}
      \vfill                          % 插入可伸缩的垂直空间,会自动填满剩余空间,用于将下面的内容“推到底部”。
      HangZhou\\
      \the\year-\ifnum\month<10 0\fi\the\month   % 判断月份是否小于 10,如果是,前面加 0,确保输出两位数格式(如 08 而不是 8)
%      \vspace{11cm}                  % 前面用了 \vfill,所以这行可能用于微调底部位置,或兼容不同页面尺寸。
    \end{center}
  \end{titlepage}
}
\makeatother