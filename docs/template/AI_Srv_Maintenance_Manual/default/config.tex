%% Overleaf			
%% Software Manual and Technical Document Template	
%% 									
%% This provides an example of a software manual created in Overleaf.




% Packages used in this example
\usepackage{graphicx}  % for including images
\usepackage{microtype} % for typographical enhancements, Provide micro layout enhancement function to improve details such as word spacing and linking
\usepackage{minted}    % for code listings
\usepackage{amsmath}   % for equations and mathematics
\setminted{style=friendly,fontsize=\small}  % for code highlight, fontsize == small

\renewcommand{\listoflistingscaption}{List of Code Listings}   %Redefine the title of the code list as' List of Code Listings'

\usepackage{hyperref}  % for hyperlinks
\usepackage[a4paper,top=2.2cm,bottom=2.2cm,left=2.5cm,right=2.5cm]{geometry} % for setting page size and margins
\usepackage{xcolor}       % color
\usepackage{listings}     % for code highlight, specially for bash command
\usepackage{enumitem}     % Provide customized list environment functionality

% Custom macros used in this example document
\newcommand{\doclink}[2]{\href{#1}{#2}\footnote{\url{#1}}}  % Define the \ doclink command to create hyperlinks with footnotes, example: \doclink{URL}{disjplay text}

\newcommand{\cs}[1]{\texttt{\textbackslash #1}}  %Define the \ cs command to display LaTeX commands

% Frontmatter data; appears on title page
\title{AI Server Maintenance \\Manual}
\version{1.0.0}
\author{Qiao Lei}
\softwarelogo{\includegraphics[width=6cm]{res/Logo.png} \vspace{6cm}}    % vspace{6cm}, Leave a vertical spacing of 6cm behind the image

% ---------code minted highlight configure---------------------------------
% line number,frame,font,backgroud,etc..... pay attention to overleaf free plan , no bash highlight, need use the package listings for bash...
\definecolor{bgcolor}{HTML}{F5F5F5}
\setminted{
    fontsize=\small,
    style=colorful,       
    linenos,
    numbersep=5pt,
    xleftmargin=1em,
    frame=lines,
    framesep=1mm,
    tabsize=4,
    baselinestretch=1.5,
    bgcolor=bgcolor        % first use \definecolor to define
}

%
\definecolor{mygreen}{RGB}{0,128,0}
\definecolor{myred}{RGB}{150,0,0}
\definecolor{mygray}{RGB}{128,128,128}
\lstdefinestyle{ShellStyle}{
  language=bash,
  keywordstyle=\color{magenta}\bfseries,
  commentstyle=\color{mygreen}\itshape,
  stringstyle=\color{myred},
  basicstyle=\small\ttfamily,
  backgroundcolor=\color{mygray!5},
  frame=single,
  numbers=left,
  numberstyle=\tiny\color{mygray},
  tabsize=4,
  breaklines=true,
  showstringspaces=false
}
\lstset{style=ShellStyle}

\newcommand{\shortdashline}[1]{%
  \noindent\makebox[\textwidth][c]{%
    \xleaders\hbox{\textemdash}\hfill
  }
}

\newcommand*{\colorline}[2][red]{%
  \par
  \noindent
  \makebox[\linewidth][c]{%
    \color{#1}\rule{\linewidth}{#2}%
  }%
  \par
}

\newcommand{\keywords}[1]{%
  \par\noindent\textbf{Keywords:}\;#1\par
}

\renewcommand{\abstractname}{\Large\bfseries Abstract}