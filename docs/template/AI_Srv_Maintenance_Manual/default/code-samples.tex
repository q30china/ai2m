Code can be printed using the minted or listings packages, or several other tools.
Listing~\ref{listing:octave} shows an example of typesetting code from an external file.

\begin{listing}
\inputminted{octave}{code/BitXorMatrix.m}
\caption{Example from external file}
\label{listing:octave}
\end{listing}

Alternatively, code can be written directly in your .tex files, as in Listing~\ref{listing:py}.
It's also possible to typeset syntax-highlighted code inline with a number of code listing packages.
Check the documentation for the package you're using for more details.
For example, section~3.3 of the \doclink{http://texdoc.net/pkg/minted}{Minted package documentation} gives some examples and guidelines, as does our help article on \doclink{https://www.overleaf.com/learn/latex/Code_Highlighting_with_minted}{code highlighting with minted}. Forums such as \doclink{https://tex.stackexchange.com/}{TeX StackExchange} and \doclink{https://latex.org/forum/}{LaTeX Community} are also a great source of tips.

\begin{listing}
\begin{minted}{python}
print("Hello World")
\end{minted}
\caption{Example Python code}
\label{listing:py}
\end{listing}


\begin{figure}[H]
    \begin{center}
        \includegraphics[width=.95\linewidth]{zabbix_qgis_page0.png}\\
        \caption{ Qgisweb.}\label{zabbix_qgis_page0}
    \end{center}
\end{figure} 


\vspace{-.7cm}
\begin{enumerate}
    \item Server address: 172.16.33.244
    \item Admin User name: Administrator, Password: hexing@2025
    \item User name : ems, Password: sme8003
\end{enumerate}

\colorline[red]{1pt}
\colorline[green!60!black]{1.5pt}
\colorline[gray]{0.1pt}
\shortdashline