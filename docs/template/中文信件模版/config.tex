% ================================
% 颜色配置区域
% ================================

% 主色调配置
\definecolor{titlecolor}{RGB}{0, 82, 155}        % 标题颜色(商务蓝)
\definecolor{subtitlecolor}{RGB}{128, 0, 0}      % 副标题颜色(深红色)
\definecolor{sendercolor}{RGB}{0, 100, 0}        % 发件人标题颜色(深绿色)
\definecolor{receivercolor}{RGB}{139, 69, 19}    % 收件人标题颜色(棕色)
\definecolor{infocolor}{RGB}{80, 80, 80}         % 信息文字颜色(深灰色)
\definecolor{contentcolor}{RGB}{0, 0, 0}         % 正文颜色(黑色)
\definecolor{signaturecolor}{RGB}{128, 0, 128}   % 签名颜色(紫色)
\definecolor{captioncolor}{RGB}{0, 0, 139}       % 图片标题颜色(蓝色)
\definecolor{linecolor}{RGB}{200, 200, 200}      % 分隔线颜色(浅灰色)

% ================================
% 信件基本信息配置
% ================================

% 信件标题和副标题
\newcommand{\lettertitle}{朋友往来信函}
\newcommand{\lettersubtitle}{—— 关于初始信件 ——}

% 发件人信息
\newcommand{\sendername}{乔磊}
\newcommand{\senderaddress}{浙江省杭州市拱墅区莫干山路1418-5 杭州海兴电力科技股份有限公司 配用电事业群 测试部}
% \newcommand{\senderphone}{138-xxxx-xxxx}
% \newcommand{\senderemail}{zhangsan@company.com}
\newcommand{\sendersignature}{乔磊} % 手写签名
\newcommand{\signaturedate}{\today} % 签名日期

% 收件人信息
\newcommand{\receivername}{钭军勇}
\newcommand{\receivernickname}{豆豆兄弟}
\newcommand{\receiveraddress}{上海市浦东新区周浦镇康沈路2077号 3134}
\newcommand{\letterdate}{\today} % 信件日期

% ================================
% 图片配置
% ================================

% 图片路径(如果不需要图片,请设为空:\newcommand{\letterimage}{})
\newcommand{\letterimage}{example-image} % 示例图片,实际使用时替换为你的图片路径
\newcommand{\imagecaption}{相关业务示意图} % 图片标题
\newcommand{\imagewidth}{0.8\textwidth} % 图片宽度比例

% ================================
% 其他配置
% ================================

% 是否显示调试网格(用于布局调整)
\newcommand{\showdebuggrid}{false} % 设为 true 可以显示调试网格线