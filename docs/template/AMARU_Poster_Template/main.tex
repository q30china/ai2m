\documentclass[portrait]{Hylangtechposter}


\begin{document}

\printheader

\begin{center}
%% TITLE
\textbf{\bf\veryHuge\color{NavyBlue}Title of the poster for the AMARU meeting \\[1.5cm]}

%% AUTHORS

\huge     First Author $^{\spadesuit}$ ~
    Second Author $^{\spadesuit}$ ~ 
    Third Author$^{\diamondsuit}$ ~ 
    F. Author$^{\spadesuit}$ \\[0.2cm] 
\Large Affiliation 1$^{\heartsuit}$ ~ 
    Affiliation 2$^{\spadesuit}$ ~
    Affiliation 3$^{\spadesuit}$ \\[0.2cm]
\Large    $^{\spadesuit}$\texttt{\{firstname.lastname\}@helsinki.fi} \qquad $^{\diamondsuit}$\texttt{thirdone@nsecondaffiliation.com} \\
\Large   $^{\heartsuit}$\texttt{noname.3affiliation@somerandomuni.edu} 
 \end{center}

\vspace{2cm}

%%%% If you have an abstract:  uncomment %%%%
%%%%
 \color{Navy}
% \begin{abstract}
%Si desea agregar un abstract del poster puede agregarlo aqui. 

%Te ancillae contentiones vix, ad partiendo patrioque inciderint eum. Mea porro regione ex. Case iuvaret ocurreret quo at, legere malorum indoctum cu his. Eros ubique mel in. At duo partem vidisse intellegam. Equidem detraxit has ea. Phasellus imperdiet, tortor
%   vitae congue bibendum, felis enim sagittis lorem, et volutpat ante
%   orci sagittis mi. Morbi rutrum laoreet semper. Morbi accumsan enim
%   nec tortor consectetur non commodo nisi sollicitudin. Proin
%   sollicitudin. Pellentesque eget orci eros. Fusce ultricies, tellus
%   et pellentesque fringilla, ante massa luctus libero, quis tristique
%   purus urna nec nibh.
  
% \end{abstract}

% \vspace{1cm}

% As FONTES podem ser aumentadas para
% \large \Large \LARGE \huge \Huge \veryHuge \VeryHuge \VERYHuge

 \Large

\begin{multicols}{2} % begin two columns

  \color{black}
  
\section*{Introduction}
%\lipsum[1]

El siguiente formato de poster es apenas indicativo. 

\begin{itemize}

    \item Este ejemplo de poster es apenas indicativo. Pudiendo ser de una o dos columnas, y con los gr\'aficos, y ecuaciones que considere apropiado.  

    \item Los logos de CYTED, AMARU (a la izquierda) as\'i como los de la UNA y FPUNA (a la dereccha) indicados en la parte superior del poster, son madatorios (de preferencia). Si desea agregar logos adicionales puede incluirlos en el centro entre ambos logos. 

    \item Si desea puede incluir un abstract puede hacerlo.

    \item El orden de los autores siguiendo la presentacion en el abstract. Las filiaciones institucionales, de la forma se expresa como ejemplo. Asi como los emails. 

    \item La estructura es indicativa, puede usar una estructura mas convencional como: Introduction - Methodology - Results and Discussion - Conclusions. 

    \item Los agradecimientos personales e insticuonales al final del texto. Los mencionados en este modelo de poster son apenas indicativos como ejemplo. 
    
    \item Se recomienda tambien agardecer a las instituciones financistas. 
    
\end{itemize}



\section*{Contributions}

% \color{DarkSlateGray} % DarkSlateGray color for the rest of the content

\begin{enumerate}
\item Lorem ipsum dolor sit amet, consectetur.
\item Nullam at mi nisl. Vestibulum est purus, ultricies cursus volutpat sit amet.
\end{enumerate}

\section{Experimental setup}

Sed ut perspiciatis, unde omnis iste natus error sit voluptatem
accusantium doloremque laudantium, totam rem aperiam eaque ipsa, quae
ab illo inventore veritatis et quasi architecto beatae vitae dicta
sunt, explicabo. Nemo enim ipsam voluptatem na Figura~1, quia voluptas
sit, aspernatur aut odit aut fugit, sed quia consequuntur magni
dolores eos.

\begin{center}\vspace{1cm}
\includegraphics[width=0.5\linewidth]{Logo_CYTED_AMARU}
\captionof{figure}{\color{Green} PLOT of Figures}
\end{center}\vspace{1cm}


\section{Experiments \& Methodology}

Fusce magna risus, molestie ut porttitor in, consectetur sed
mi. Vestibulum ante ipsum primis in faucibus orci luctus et ultrices
posuere cubilia Curae; Pellentesque consectetur blandit
pellentesque. Sed odio justo, viverra nec porttitor vel, lacinia a
nunc. Suspendisse pulvinar euismod arcu, sit amet accumsan enim
fermentum quis. In id mauris ut dui feugiat egestas. 
\begin{equation}
E = mc^{2}
\label{eqn:Einstein}
\end{equation}

Curabitur mi sem, pulvinar quis aliquam rutrum. (1) edf (2) ,
$\Omega=[-1,1]^3$, maecenas leo est, ornare at. $z=-1$ edf $z=1$ sed
interdum felis dapibus sem. $x$ set $y$ ytruem.  Turpis $j$ amet
accumsan enim $y$-lacina; ref $k$-viverra nec porttitor $x$-lacina.

Vestibulum ac diam a odio tempus congue. Vivamus id enim nisi:

\begin{eqnarray}
\cos\bar{\phi}_k Q_{j,k+1,t} + Q_{j,k+1,x}+\frac{\sin^2\bar{\phi}_k}{T\cos\bar{\phi}_k} Q_{j,k+1} &=&\nonumber\\ 
-\cos\phi_k Q_{j,k,t} + Q_{j,k,x}-\frac{\sin^2\phi_k}{T\cos\phi_k} Q_{j,k}\label{edgek}
\end{eqnarray}
and
\begin{eqnarray}
\cos\bar{\phi}_j Q_{j+1,k,t} + Q_{j+1,k,y}+\frac{\sin^2\bar{\phi}_j}{T\cos\bar{\phi}_j} Q_{j+1,k}&=&\nonumber \\
-\cos\phi_j Q_{j,k,t} + Q_{j,k,y}-\frac{\sin^2\phi_j}{T\cos\phi_j} Q_{j,k}.\label{edgej}
\end{eqnarray} 

Donec faucibus purus at tortor egestas eu fermentum dolor
facilisis. Maecenas tempor dui eu neque fringilla rutrum. Mauris
\emph{lobortis} nisl accumsan. Aenean vitae risus ante.
%

\vspace{1cm}
\begin{center}
\begin{tabular}{l l l}
\toprule
\textbf{Treatments}~ & \textbf{Response 1} & \textbf{Response 2}\\
\midrule
Treatment 1 & 0.0003262 & 0.562 \\
Treatment 2 & 0.0015681 & 0.910 \\
Treatment 3 & 0.0009271 & 0.296 \\
\bottomrule
\end{tabular}
\captionof{table}{\color{Green} Table caption}
\end{center}
\vspace{1cm}

Phasellus imperdiet, tortor vitae congue bibendum, felis enim sagittis
lorem, et volutpat ante orci sagittis mi. Morbi rutrum laoreet
semper. Morbi accumsan enim nec tortor consectetur non commodo nisi
sollicitudin. Proin sollicitudin.

\section{Conclusions || Discussion }

O manual do \TeX~\cite{knuth1986} pode ser usado para aprendê-lo, e o livro do
Lamport~\cite{lamport1994} para aprender o \LaTeX, mas se quiser ir a fundo tem
que ver como o \TeX quero o os parágrafos em linhas~\cite{knuth1981}.


\begin{itemize}
\item Pellentesque eget orci eros. Fusce ultricies, tellus et
  pellentesque fringilla, ante massa luctus libero, quis tristique
  purus urna nec nibh. 
\end{itemize}

\bibliographystyle{plain} % Plain referencing style
\bibliography{refs} % Use the example bibliography file %sample.bib

%----------------------------------------------------------------------------------------
%   AGRADECIMENTOS
%----------------------------------------------------------------------------------------
\section*{Acknowledgements}
\small \textnormal{This work was partially supported by FEEI-CONACYT-PROCIENCIA-PINVYYY\textnumero{}~771113. ~We ~also ~thank ~the CYTED, for supporting the traveling.}

% por favor que lo de abajo este como ultima linea
%{\bf Powered by the Group on Research in Scientific Computing and Applied Mathematics (SC\&MA), Polytechnic School, UNA.  } 




\end{multicols}

\end{document}