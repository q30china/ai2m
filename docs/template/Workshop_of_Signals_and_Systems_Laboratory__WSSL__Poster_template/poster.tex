\documentclass[isoft]{ssltexposter}

\usepackage{lipsum}
\usepackage{natbib}
\usepackage{booktabs}
\usepackage{subfig} 
\usepackage{amsmath} 
\usepackage{textcomp} 
\usepackage{url}  
\usepackage[hidelinks]{hyperref}
\usepackage[utf8]{inputenc}
\usepackage[brazil, english]{babel}

%%%%%%%%%%%%%%%%%%%%%%%%%%%%%%%%%%%%%%%%%
%%               Configs               %%
%%%%%%%%%%%%%%%%%%%%%%%%%%%%%%%%%%%%%%%%%

% Choose one of the section color {ufglhblue | ufgdkblue | dkblue | black | gold}
\setsectioncolor{green!50!black}

% Define width of the rule or hide it by setting 0pt or commenting the command 
\setcolumnseprule{0pt}

% Inform the paths to the logo files or leave empty one or both parameters. 
% There are three options [ T | M | B ] to positioning them.
\setlogos[T]{images/logotipo-ufabc-abaixo.eps}{images/lss_logo_font-white_signal-white.pdf}

% Choose one of the background options {1 | 2 | 3}. 
% Actually, one can select any graphic file in backgrounds directory. 
\setbackground{1c}

% Resize the title to keep it in two lines // {font size}{line height}
\settitlesize{64pt}{68pt}

% Resize the font of the content. Default {32pt}{38pt} // {font size}{line height}
\setcontentfontesize{32pt}{40pt}

% Resize the font of the emails. Default {26pt}{32pt} // {font size}{line height}
\setemailfontesize{42pt}{40pt}

% General info
\title{\uppercase{TITLE}} 

\author{Author 1, Author 2, ... , Author N} 

% \department{Laboratório de Sinais e Sistemas\\Universidade Federal do ABC (UFABC)}

\email{ \text{author1, author2, ... , authorN}@aluno.ufabc.edu.br \\\text{author1, author2, ... , authorN}@ufabc.edu.br }

% \class{I Workshop do Laboratório de Sinais e Sistemas (WLSS)}

% \year{2019}

\conference{II Workshop of Signals and Systems Laboratory (WSSL 2019)}

%%%%%%%%%%%%%%%%%%%%%%%%%%%%%%%%%%%%%%%%%
%%           End configs               %%
%%%%%%%%%%%%%%%%%%%%%%%%%%%%%%%%%%%%%%%%%

\pagestyle{fancy}
\begin{document}
    \begin{poster}
    %%%%%%%%%%%%%%%%%%%%%%%%%%%%%%%%%%%%%%%%%
    %%             Begin poster            %%
    %%%%%%%%%%%%%%%%%%%%%%%%%%%%%%%%%%%%%%%%%
    
    \begin{abstract}
        \normalsize
        
        \lipsum[57]
        
    \end{abstract}
    
    \section{Introduction}
    
        \lipsum[57]
        
    \section{Methodology}%
        
        \subsection{Subsection I}
        
        \lipsum[57]
        
        \vspace{1cm}
        
        \begin{figure}
            \centering
            \captionsetup{type=figure}
            \includegraphics[scale=1.5]{example-image-a}
            \caption{Example of a figure.}
            \label{fig:lstm}
        \end{figure}
        
        \subsection{Subsection II}
        
            \lipsum[57]
            
            \vspace{1cm}
            \begin{table}
                \centering
                \captionsetup{type=table}
                \caption{Example of a table.}
                \label{Corpus}
                \renewcommand{\arraystretch}{1.2}
                \resizebox{0.47\textwidth}{!}{%
                \begin{tabular}{ccccccccccc}
                    \toprule
                    & \textbf{Column 1} &  &  \textbf{Column 2} &  &  \textbf{Column 3} & &  \textbf{Column 4} &   & \textbf{Column 5} & \\
                    \midrule
                    & xxxxx & & xxxxx & & xxxxx & & xxxxx & & xxxxx &\\ 
                    & yyyyy & & yyyyy & & yyyyy & & yyyyy & & yyyyy &\\ 
                    & zzzzz & & zzzzz & & zzzzz & & zzzzz & & zzzzz &\\ 
                    \bottomrule
                \end{tabular}
            }
            \end{table}
            \vspace{1cm}
            
            \lipsum[57]
            
            \citep{defects2j}.
    
            \subsection{Subsection III}
                \lipsum[57]
    
        \section{Results}%
                
            \lipsum[57]  
    
            \begin{figure}
                \centering
                \captionsetup{type=figure}
                \includegraphics[scale=1.5]{example-image-b}
                \caption{Another example of a figure.}
                \label{fig:result}
            \end{figure}
    
            \lipsum[57]
            \cite{chollet2015keras}     
    
        \section{Conclusion}
    
            \lipsum[57]
    
        \section{Acknowledgment}
    
            \lipsum[57]
    
        \bibliographystyle{abbrv}
        \bibliography{refs}
    
    %%%%%%%%%%%%%%%%%%%%%%%%%%%%%%%%%%%%%%%%%
    %%               End poster            %%
    %%%%%%%%%%%%%%%%%%%%%%%%%%%%%%%%%%%%%%%%%
    
    \end{poster}

\end{document}

 
